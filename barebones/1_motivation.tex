%-TG: past motivation for linear-rnn models might not hold anymore.
The motivation for linear-time models may be less compelling today, given the efficiency advancements in Transformers.
%
Are Transformers the endgame?\footnote{Sasha's ``Transformers in 2027'' bet: \url{https://www.isattentionallyouneed.com}.} I think not.
%
Are linear models? I'm unsure!

%-TG: motivate the need for better models.
As we become increasingly hardware-aware in advancing Transformers---through techniques like Flash-$\ast$ algorithms \citep{fu_flashfftconv_2023,shah_flashattention-3_2024} and compute-optimal training \citep{muennighoff_scaling_2023}---the previously prohibitive quadratic complexity of Transformers has become less of a limitation for pretraining.
%
The real ``brick wall,'' however, lies in high inference costs, especially for the latest Transformers like OpenAI o$1$, which may be employing implicit chain-of-thought (CoT) with stepwise internalization.
%
Even with speculative decoding and KV-cache optimizations, relying on attention for very large contexts (e.g., $100$M tokens\footnote{LTM-$2$-mini handles $100$M context windows through a proprietary \textit{sequence-dimension algorithm} that may not rely entirely on attention: \url{https://magic.dev/blog/100m-token-context-windows}.}) is infeasible.

Distilling pretrained Transformers to linear-recurrence models \citep{zhang_hedgehog_2024,wang_mamba_2024,zhang_lolcats_2024,bick_transformers_2024} seems promising.
%
However, it is uncertain if these distilled models retain the weaknesses of their base models, particularly in handling long contexts.
%
Moreover, the expressiveness of linear-recurrence models is limited by their fixed-size hidden memory \citet{jelassi_repeat_2024}.
%
The recently proposed Based model \citep{arora_simple_2024} builds on prior efforts by combining linear attention for larger hidden states with sliding-window attention for associative recall.
%
However, further evaluation is needed to assess its performance in long-context retrieval tasks (e.g., needle-in-a-haystack, hash-hop\footnote{HashHop long-context evaluation: \url{https://github.com/magicproduct/hash-hop}.}, etc.).
%
In our experiments with needle-in-a-haystack evaluations, moderately-sized models, such as RecurrentGemma-$9$B (with linear recurrence using a hidden state of $4,096$, and a local attention window of $2,048$ tokens), often struggle beyond $4\times$ the local attention window.
%
To this end, we need to devise mechanisms that reduce token storage redundancy (unlike Transformers) that support dynamic memory scaling.

%-TG: what I wish to do.
As a part of my \thedegree at \thecollegeabbr, I plan to focus on:
\begin{enumerate}[itemsep=-2pt, topsep=-1pt]
    \item analyzing the Pareto frontier of the context length vs. associative recall across varying hidden-state capacities, and
    
    \item developing models to push this frontier to achieve efficient and reliable long-context modeling.
\end{enumerate}
Designing new architectures will require understanding the limitations of current models. For instance, in hybrid models that combine attention and recurrence, it is essential to explore what long-context information persists across local attention boundaries. How would this change if the model could anticipate specific information needs in advance?\footnote{This is akin to adding a ``remember this'' to the prompt, instructing the model to retain specific information.}

\par