%-TG: from TA-ing NLP (fa22, fa23)
%-TG: teaching ML (sp25)

\paragraph{Teaching experiences.}
% 
Teaching has been instrumental in shaping my research goals and career aspirations.
% 
As a Teaching Assistant, I not only led lectures on topics like NLP in healthcare\furl{https://slides.com/tushaargangavarapu/nlp4h} and gradient-based optimization\furl{https://tinyurl.com/backprop-lec-notes} but also co-designed assignments that bridged cutting-edge research with student learning.
% 
For instance, I developed an assignment on the Seagull LM,\furl{https://tinyurl.com/seagull-lm} a small Transformer++ model that generates humorous captions using scene descriptions from the New Yorker caption contest.
%
To enhance learning outcomes, we introduced an innovative evaluation metric---the log-odds ratio between student outputs and our reference model---fostering a collaborative learning environment focused on understanding core principles rather than blind competition. 
% 
This approach reflects my commitment to making NLP research accessible and impactful.

Managing teams of teaching assistants further honed my leadership and communication skills.
% 
By coordinating a team of seven TAs, I ensured consistent quality across deliverables while creating an inclusive environment that encouraged experimentation and diverse perspectives. 
% 
These experiences have strengthened my dedication to combining research and teaching, empowering students to explore complex ideas and equipping them with the tools to contribute meaningfully to the field.